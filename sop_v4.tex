\documentclass{article}
\usepackage[letterpaper,margin=1in]{geometry}
\usepackage{xcolor}
\usepackage{fancyhdr}
\usepackage{tgschola} % Font Package

\pagestyle{fancy}
\fancyhf{}
\fancyhead[C]{%
  \footnotesize\sffamily
  \yourname\quad
  web: \textcolor{blue}{\itshape\yourweb}\quad
  \textcolor{blue}{\youremail}}

\newcommand{\soptitle}{Statemen,t of Purpose}
\newcommand{\yourname}{George Mathew}
\newcommand{\youremail}{george.meg91@gmail.com}
\newcommand{\yourweb}{georgevmathew.com}

\newcommand{\statement}[1]{\par\medskip
  \underline{\textcolor{blue}{\textbf{#1:}}}\space
}

\usepackage[
  colorlinks,
  breaklinks,
  pdftitle={\yourname - \soptitle},
  pdfauthor={\yourname},
  unicode
]{hyperref}

\begin{document}

\begin{center}\LARGE\soptitle\\
\large of \yourname\ (CSC PhD applicant Fall-2016)
\end{center}

\hrule
\vspace{1pt}
\hrule height 1pt

\bigskip

I, George Mathew am a masters student from North Carolina State University(NCSU) in Raleigh. Research and academia has always fascinated me and after a deep introspection, I have realized that my greater ambition lies in research and would want to pursue a PhD in Computer Science

\bigskip
Currently, I work as a graduate research assistant in Real-world Artificial Intelligence for 
Software Engineering(RAISE) under the supervision of Dr Tim Menzies.
 My responsibilities primarily lie in applying machine learning algorithms on software engineering 
 problems like effort estimation and utilizing optimization techniques to solve multi-objective problems software engineering models in Requirements Engineering and Software Quality Prediction. My experience with the RAISE group inspired me to pursue my career in academia and research. 

\bigskip

While pursuing my undergraduate degree, my fascination towards robotics motivated me to choose Electronics as my field of specialization. I was part of the robotics society of my university. We were actively involved in creating path follower robots and other such models. Overtime I started developing a keen interest towards image 
processing, pattern recognition and machine learning during my junior year. I worked on 
graphic projects like digital watermark remover for images and rainfall prediction using techniques like neural 
networks, support vector machines and decision trees. During my senior year I designed and implemented a digital sphygmomanometer. The project introduced me to challenges involved in integrating hardware with software. Since I was trying to solve a very specific problem of capturing signals at very low amplitudes, a customized piece of software was required to solve the challenges of efficiency and stability. After numerous iterations, I zeroed upon the final version, which as adjuded as the best project among over 100 other projects. I graduated from university capping a department gold 
medal and a university silver medal for my academic accomplishments.

\bigskip

After graduation, I started working as a Software Engineer (level-3) at Payoda Technologies. I was part of a team involved in design and development of a software load balancer called AppViewX. AppViewX is an application centric infrastructure management tool that provides visibility, analytics, security and automation to the network infrastructure. I implemented the load balancer statistics dashboard for AppViewX which collected device statistics on an hourly basis and aggregated over them. Once the statistics were processed, we were able to run predictive analysis over the data to predict faults. My short stint at Payoda Technologies has given me invaluable practical experience in the field of machine learning , 
distributed computing and network traffic management. I had the opportunity to interact and 
work with few of the best architects and managers thereby giving me the technical and business
understanding required to develop a software product.Within 11 months of my professional career I was also adjudged as the Rookie of the year 2012 and also promoted by two levels to Software Engineer (level-1). 

\bigskip
After working in a service based ecosystem for 15 months, I moved to a product based startup called CrowdChat. CrowdChat is a hashtag based platform that enabled users from different social networks to engage in ``meaningful" conversations. I was part of the  core team which developed the platform using a JavaScript and redis(NoSQL) stack. This experience introduced me to startup culture and the ability to come up with solutions and address real world problems at a rapid pace. Eventually after two years of industrial experience, I decided to pursue a masters degree to broaden my spectrum of knowledge and experience a research and academic lifestyle.


\bigskip
I have been working with Dr Menzies for over a year. I am funded by NASA Jet Propulsion Laboratory to build a Software Effort Estimator for their space programs. For this project, I collaborated with Dr Jairus Hihn, a senior systems engineer and faculty member of USC and CalTech. The project gave me great insights on how software systems play an important role in space programs and helped me collaborate with an expert, who could use his expertise to validate my experiments. This allowed me to explore territories in my research like Delphi based learning techniques. Our work was published at the NASA Cost Symposium 2015 and garnered great interest. As an extension to software effort estimation, I have worked with numerous statistical non-parametric techniques to rank the estimators used in the project.

\bigskip
During summer 2015, I interned at Facebook as a Software Development Engineeer. This experience gave me a holistic view on how to mission critical projects and helped me enhance my understanding on big data analytics. While at Facebook, I worked on their open source tools like Hive and Presto. I applied machine learning techniques like CART and Random Forests from my research to predict statistics of HDFS clusters which aided the team in monitoring the cluster status. The ``hackathon" culture at Facebook gave me a platform to collaborate with other engineers working on different fronts across the organization to implement new ideas. This experience helped me develop my team work and adaptability. 

\bigskip
Since fall 2015, I am exploring various optimization techniques. This is part of my masters thesis. Multiobjective problems have always been hard to optimize and software engineering contains many such use cases. My current research focuses on optimizing softgoals in requirements engineering. This problem is challenging since the current literature does not solve this problem sufficiently. User feedback in resolving conflicts amongst soft goals is a one such challenge, which can be addressed my enhancing the existing ``i*" framework. I plan to take a stochastic approach to traverse the model, rather than a deterministic approach. This would allow us to obtain a range of feasible solutions, which can be optimized upon based on the user requirement.

\bigskip
Apart from academics, I am involved with a few pet projects. I developed an online bookmark manager ``region.io", which allows users to store bookmarks and search through them. This was built on top of Apache Lucene which is used as an indexer for quick retrieval and seemless experience. The site has garnered over 500 users. I have also built a movie recommender system called ``octorater" which suggests movies based on IMDB ratings. The recommender system also incorporates feedback based on a simple Naive Bayes classifier. I also have experience as a free lance web developer.

\bigskip
After my PhD, I see myself as an academician and researcher. I believe best research comes from a combination of industrial and research experience. I hope to use my knowledge to contribute actively to the research community. I strongly feel the training would help me enhance my mentorship skills and excel in academia.

\end{document}
\documentclass{article}
\usepackage[letterpaper,margin=1in]{geometry}
\usepackage{xcolor}
\usepackage{fancyhdr}
\usepackage{tgschola} % Font Package

\pagestyle{fancy}
\fancyhf{}
\fancyhead[C]{%
  \footnotesize\sffamily
  \yourname\quad
  web: \textcolor{blue}{\itshape\yourweb}\quad
  \textcolor{blue}{\youremail}}

\newcommand{\soptitle}{Statement of Purpose}
\newcommand{\yourname}{George Mathew}
\newcommand{\youremail}{george.meg91@gmail.com}
\newcommand{\yourweb}{georgevmathew.com}

\newcommand{\statement}[1]{\par\medskip
  \underline{\textcolor{blue}{\textbf{#1:}}}\space
}

\usepackage[
  colorlinks,
  breaklinks,
  pdftitle={\yourname - \soptitle},
  pdfauthor={\yourname},
  unicode
]{hyperref}

\begin{document}

\begin{center}\LARGE\soptitle\\
\large of \yourname\ (CSC PhD applicant Fall-2016)
\end{center}

\hrule
\vspace{1pt}
\hrule height 1pt

\bigskip

I, George Mathew am a masters student from North Carolina State University(NCSU) in Raleigh. Research and academia has always fascinated me and after a deep introspection, I have realized that my greater ambition lies there and would want to pursue a PhD in Computer Science

\bigskip
Currently, I am a graduate research assistant of the Real-world Artificial Intelligence for 
Software Engineering(RAISE) apart from my masters program under the guidance of Dr Tim Menzies.
 My responsibilities range from applying machine learning algorithms on software engineering 
 problems like effort estimation to using optimization techniques on software engineering models
 like Requirements Engineering and CoCoMo. My research experience with the RAISE group inspired me to
 pursue my career in academia and research. 

\bigskip

I took Electronics and Instrumentation as my major for an engineering undergraduate
degree due to my facination towards robotics. I developed path follower robots and similar
miniature hardware models. I gradually started developing a keen interest towards image 
processing, pattern recognition and machine learning during my junior year. I worked on 
projects like digital watermark remover for images and rainfall prediction based on neural 
networks, support vector machines and decision trees. During my senior year as part of 
my major project I had designed and implemented a digital sphygmomanometer. The project
gave me a great insight into integrating hardware and software. It required the customized 
filter software to be efficient and highly stable. After iteratively improvising the design I 
arrived at an optimal software that integrates with the sensors. The project was eventually 
judged as the best project of the year. I graduated from college capping a department gold 
medal and a university silver medal for my academic accomplishments.

\bigskip

I started working as a Software Engineer (level-3) in PayodaTechnologies right after under 
graduation because of their then ongoing work on big data analytics. My short stint at Payoda
Technologies has given me invaluable practical experience in the field of machine learning , 
distributed computing and network traffic management. I had the opportunity to interact and 
work with few of the best architects and managers thereby giving me the technical and business
understanding required to develop a software product. During my period at Payoda Technologies
I worked with technologies like Java, JavaScript and on SQL and NoSQL based 
databases.This broadened my technical knowledge and greatly developed my programming 
skills. Within 11 months of my professional career I was also adjudged as the Rookie 
of the year 2012 and also promoted by two levels to Software Engineer (level-1). I then moved 
to CrowdChat which was a hashtag based platform that enabled users from different social 
networks to engage in meaningful conversations. I was part of the  core team that developed 
the entire platform on a JavaScript and redis(NoSQL) stack. It gave me a great understanding  of startup
culture and the ability to come up with solutions and address real world problems at a rapid pace. Eventually
after two years of industrial exposure, I decided to pursue my masters to broaden my spectrum
and experience a research and academic lifestyle.


\bigskip
I started working with Dr Tim Menzies from the second semester of my masters program. I was funded by NASA Jet Propulsion Laboratory on building a Software Effort Estimator for their various space programs. I was guided by Dr Jairus Hihn, a senior systems engineer and faculty member of USC and CalTech. The project gave me a great insight on how software systems play a serious role in space programs and gave me the oppurtunity to validate my machine learning techniques with an expert. This allowed me to explore territories in my research I would not have been able to work directly like Delphi based learning techniques. Our work was published at the NASA Cost Symposium and garnered great interest due to our results. Apart from machine learning and software engineering, I also had the oppurtunity to work with various statistical non parametric estimation methods like cliffs-delta and a12 test. This was an extension to our effort estimation techniques where statistical tests were used to rank our estimators.

\bigskip
During summer 2015, I had the privilege to work as a Software Engineering Intern at Facebook. This gave me a holistic approach how mission critical projects are managed and also gave me a greater insight into big data analytics. I was working on facebook's own metastore tools Hive and Presto. I was able to apply machine learning techniques like CART and Random Forests from my research in statistics prediction across Facebook datacenters. Since Fall 2015 I started working on optimization algorithms. Multiobjective problems have always been hard to optimize and software engineering contains many such use cases. My current research is on optimizing softgoals in requirements engineering. This problem is really challenging since there is not much active research in the community in this front. Moreover, the problem statement is very abstract and there can be lot of ambiguities due to conflicting softgoals.

\bigskip
Apart from academics, I also work on a few pet projects. I developed an online bookmark manager "region.io", that allows a user to store bookmarks and search through them. This was built on top of Apache Lucene which was used as an indexer for quick retrieval and seemless experience. The site has garnered around 500 users. I have also built a movie recommender system called "octorater" which suggests movies based on IMDB ratings. I was able to incorporate a feedback based recommender using a simple Naive Bayes classifier. Web development is another front that has always interested me. I have built a few responsive websites for my friends and an NGO in India. 

\bigskip
After my PhD, I see myself as an academician and researcher. I hope to use my knowledge to contribute actively in the research fraternity. I further wish the arduous experience can cater to enhance my skills in mentoring students and excel in academia.

\end{document}